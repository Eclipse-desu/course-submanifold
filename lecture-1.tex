\documentclass{ctexart}

\usepackage{amsmath}
\usepackage{amssymb}
\usepackage{amsthm}

\usepackage{mathrsfs}

\newcommand{\bnabla}{\overline{\nabla}}
\newcommand{\tnabla}{\widetilde{\nabla}}
\newcommand{\VecFld}{\mathscr{X}}
\newcommand{\Rm}{\mathrm{Rm}}
\newcommand{\bRm}{\overline{\Rm}}

\newcommand{\Div}[2]{\nabla_{#1}{#2}}
\newcommand{\bDiv}[2]{\bnabla_{#1}{#2}}
\newcommand{\tDiv}[2]{\tnabla_{#1}{#2}}

\newcommand{\Shape}[1]{A_{#1}}

\newcommand{\Inner}[2]{\langle {#1}, {#2} \rangle}

\newtheorem{theorem}{定理}[section]
\newtheorem{lemma}{引理}
\newtheorem{definition}{定义}

\begin{document}
\section{子流形的基本公式}
\subsection{等距浸入和嵌入}
\begin{definition}[等距浸入]
	若映射 $F\colon (M^n, g) \hookrightarrow (N^{n + p}, \bar{g})$ 是浸入, 且 $M$ 上的度量恰为由 $F$ 将 $N$ 上度量拉回得到的, 即 $g = F^{\ast}\bar{g}$, 则称 $F$ 是{\bf 等距浸入}.
\end{definition}

\begin{definition}[嵌入]
	若映射 $F\colon (M^n, g) \hookrightarrow (N^{n + p}, \bar{g})$ 是浸入, 且还满足
	\begin{enumerate}
		\item $F$ 是单射;
		\item $F\colon M \rightarrow F(M)$ 是同胚;
	\end{enumerate}
	则称 $F$ 是{\bf 嵌入}.
\end{definition}

若 $F \colon M \rightarrow N$ 同时是等距浸入和嵌入, 则称其是{\bf 等距嵌入}, 同时可将 $N$ 的子集 $F(M)$ 和 $M$ 等同, $F(M)$ 上的度量即为包含映射 $\iota \colon F(M) \rightarrow N$ 诱导的度量.

以下除特殊声明外, $M^n$ 是 $N^{n + p}$ 的嵌入子流形.

\begin{definition}
	设 $x \in M \subset N$, 则 $T_xM \subset T_xN$, 其正交补记作 $N_xM$. $NM = \cup_{x \in M}N_xM$ 为 $M$ 的{\bf 法丛}, $N_xM$ 为 $M$ 在 $x$ 点处的{\bf 法空间}. 由定义 $T_xN = T_x{M} \oplus N_xM$.
\end{definition}

设 $\bar{\nabla}$ 是 $N$ 上和 $\bar{g}$ 相容的黎曼联络, 取 $M$ 上向量场 $X, Y \in \VecFld(M)$. 可以将 $X, Y$ 延拓到 $N$ 上, 得到 $N$ 上的切向量场 $\bar{X}, \bar{Y}$. 则

$$
	\bar{\nabla}_{X}Y := \left.\bar{\nabla}_{\bar{X}}\bar{Y}\right|_{M}
$$
同时这个限制和 $X$ 以及 $Y$ 的延拓方式无关. $\bar{\nabla}_{X}Y$ 是 $T_xN$ 中的向量, 对其按切向和法向分解:

\begin{equation}\label{eq::Gauss-formula}
	\bar{\nabla}_{X}Y = \nabla_XY + h(X, Y)
\end{equation}
其中 $\nabla_XY \in T_xM$ 是切向部分, 且 $\nabla$ 是 $M$ 上和 $g$ 相容的黎曼联络. 而 $h$ 是一个对称函数双线性 
\end{document}