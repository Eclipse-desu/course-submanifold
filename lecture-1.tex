\documentclass{ctexart}

\usepackage{amsmath}
\usepackage{amssymb}
\usepackage{amsthm}

\newtheorem{theorem}{定理}[section]
\newtheorem{lemma}{引理}
\newtheorem{definition}{定义}

\begin{document}
\section{子流形的基本公式}
\subsection{等距浸入和嵌入}
\begin{definition}[等距浸入]
	若映射 $F\colon (M^n, g) \hookrightarrow (N^{n + p}, \bar{g})$ 是浸入, 且 $M$ 上的度量恰为由 $F$ 将 $N$ 上度量拉回得到的, 即 $g = F^{\ast}\bar{g}$, 则称 $F$ 是{\bf 等距浸入}.
\end{definition}
\begin{definition}[嵌入]
	若映射 $F\colon (M^n, g) \hookrightarrow (N^{n + p}, \bar{g})$ 是浸入, 且还满足
	\begin{enumerate}
		\item $F$ 是单射;
		\item $F\colon M \rightarrow F(M)$ 是同胚;
	\end{enumerate}
	则称 $F$ 是{\bf 嵌入}.
\end{definition}
\begin{definition}
\end{definition}
\end{document}