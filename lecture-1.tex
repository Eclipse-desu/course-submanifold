\documentclass{ctexart}

\usepackage{amsmath}
\usepackage{amssymb}
\usepackage{amsthm}

\usepackage{mathrsfs}

\newcommand{\bnabla}{\overline{\nabla}}
\newcommand{\tnabla}{\widetilde{\nabla}}
\newcommand{\VecFld}{\mathscr{X}}
\newcommand{\Rm}{\mathrm{Rm}}
\newcommand{\bRm}{\overline{\Rm}}

\newcommand{\Div}[2]{\nabla_{#1}{#2}}
\newcommand{\bDiv}[2]{\bnabla_{#1}{#2}}
\newcommand{\tDiv}[2]{\tnabla_{#1}{#2}}

\newcommand{\Shape}[1]{A_{#1}}

\newcommand{\Inner}[2]{\langle {#1}, {#2} \rangle}

\newtheorem{theorem}{定理}[subsection]
\newtheorem{lemma}[theorem]{引理}
\newtheorem{definition}[theorem]{定义}

\begin{document}
\section{子流形的基本公式}
\subsection{等距浸入和嵌入}
\begin{definition}[等距浸入]
	若映射 $F\colon (M^n, g) \hookrightarrow (N^{n + p}, \bar{g})$ 是浸入, 且 $M$ 上的度量恰为由 $F$ 将 $N$ 上度量拉回得到的, 即 $g = F^{\ast}\bar{g}$, 则称 $F$ 是{\bf 等距浸入}.
\end{definition}

\begin{definition}[嵌入]
	若映射 $F\colon (M^n, g) \hookrightarrow (N^{n + p}, \bar{g})$ 是浸入, 且还满足
	\begin{enumerate}
		\item $F$ 是单射;
		\item $F\colon M \rightarrow F(M)$ 是同胚;
	\end{enumerate}
	则称 $F$ 是{\bf 嵌入}.
\end{definition}

若 $F \colon M \rightarrow N$ 同时是等距浸入和嵌入, 则称其是{\bf 等距嵌入}, 同时可将 $N$ 的子集 $F(M)$ 和 $M$ 等同, $F(M)$ 上的度量即为包含映射 $\iota \colon F(M) \rightarrow N$ 诱导的度量.

以下除特殊声明外, $M^n$ 是 $N^{n + p}$ 的嵌入子流形.

\subsection{法空间·法丛}
\begin{definition}
	设 $x \in M \subset N$, 则 $T_xM \subset T_xN$, 其正交补记作 $N_xM$. $NM = \cup_{x \in M}N_xM$ 为 $M$ 的{\bf 法丛}, $N_xM$ 为 $M$ 在 $x$ 点处的{\bf 法空间}. 由定义 $T_xN = T_x{M} \oplus N_xM$.
\end{definition}

\subsection{第二基本形式·平均曲率向量}
设 $\bnabla$ 是 $N$ 上和 $\bar{g}$ 相容的黎曼联络, 取 $M$ 上向量场 $X, Y \in \VecFld(M)$. 可以将 $X, Y$ 延拓到 $N$ 上, 得到 $N$ 上的切向量场 $\bar{X}, \bar{Y}$. 则
\begin{equation}\label{eq::restriction-of-bar-nabla}
	\bnabla_{X}Y := \left.\bnabla_{\bar{X}}\bar{Y}\right|_{M},
\end{equation}
同时这个限制和 $X$ 以及 $Y$ 的延拓方式无关. $\bnabla_{X}Y$ 是 $T_xN$ 中的向量, 对其按切向和法向分解:
\begin{equation}\label{eq::Gauss-formula}
	\bnabla_{X}Y = \nabla_XY + h(X, Y),
\end{equation}
其中 $\nabla_XY \in T_xM$ 是切向部分, 且 $\nabla$ 是 $M$ 上和 $g$ 相容的黎曼联络. 而 $h \colon \VecFld(M) \times \VecFld(M) \rightarrow N(M)$ 是一个对称双线性张量场. 公式 \ref{eq::Gauss-formula} 称为{\bf Gauss 公式}. 对于 $h$ 有如下定义:

\begin{definition}
	$h$ 为子流形 $M$ 上的{\bf 第二基本形式}. 借助它可以定义法向量场
	\begin{equation}\label{eq::mean-curvature-vector-def}
		\overrightarrow{H} := \dfrac{1}{n}\mathrm{tr}_{g}h,
	\end{equation}
	即 $h$ 迹的平均, 为子流形 $M$ 的{\bf 平均曲率向量}.
\end{definition}

\subsection{Weingarten 公式·形状算子}
假设如上, 但考虑光滑法向量场 $\xi \in N(M)$ 及其延拓 $\bar{\xi} \in \VecFld(N)$. 类似有:
\begin{equation*}
	\bDiv{X}{\xi} := \left.\bDiv{\bar{X}}{\xi}\right|_{M},
\end{equation*}
这个延拓也与 $X$ 以及 $\xi$ 的延拓方式无关. $\bDiv{X}{\xi}$ 是 $T_xN$ 中的向量, 对其按切向和法向分解:
\begin{equation}\label{eq::Weingarten-formula}
	\bDiv{X}{\xi} = -\Shape{\xi}X + \tDiv{X}{\xi},
\end{equation}
其中 $\Shape{\xi} \colon \VecFld(M) \rightarrow \VecFld(M)$ 是线性变换, 称为{\bf 形状算子}或 {\bf Weingarten 变换}. $\tnabla \colon \VecFld(M) \times N(M) \rightarrow N(M)$ 是{\bf 法丛的联络}.

\subsection{第二基本形式和形状算子的关系}
取切向量场 $X, Y \in \VecFld(M)$, 法向量场 $\xi \in N(M)$. 则可以得到如下等式:
\begin{equation*}
	0 = X\Inner{Y}{\xi} = \Inner{\bDiv{X}{Y}, \xi} + \Inner{Y}{\bDiv{X}{\xi}} = \Inner{h(X, Y)}{\xi} + \Inner{Y}{-\Shape{\xi}X}.
\end{equation*}
因此得到等式:
\begin{equation}
	\Inner{h(X, Y)}{\xi} = \Inner{Y}{\Shape{\xi}{X}}
\end{equation}

\subsection{子流形基本方程·序}
下面推导子流形上的基本方程.

依然设 $X, Y$ 是 $M$ 上光滑向量场, $\bar{X}, \bar{Y}$ 是其在 $N$ 上的光滑延拓. 则在 $M$ 上, Lie 括号 $\left.[\bar{X}, \bar{Y}]\right|_{M} = [X, Y]$ 是和延拓方式无关的. 因此结合 \ref{eq::restriction-of-bar-nabla} 式, 接下来在 $M$ 上可将 $X, Y$ 和 $\bar{X}, \bar{Y}$ 等同.

设 $\bRm$ 为 $N$ 上的曲率张量, $\Rm$ 为 $M$ 上的曲率张量. 再取子流形上的向量场 $Z$,
\begin{align*}
	&\bRm(X, Y)Z \\
	&= \bDiv{X}{\bDiv{Y}{Z}} - \bDiv{Y}{\bDiv{X}{Z}} - \bDiv{[X, Y]}{Z} \\
	&= \bDiv{X}{\left(\Div{Y}{Z} + h(Y, Z)\right)} - \bDiv{Y}{\left(\Div{X}{Z} + h(X, Z)\right)} - \left(\Div{[X, Y]}{Z} + h([X, Y], Z)\right) \\
	&= \Div{X}{\Div{Y}{Z}} + h(X, \Div{Y}{Z}) - \Shape{h(Y, Z)}{X} + \tDiv{X}{h(Y, Z)} \\
	&- \Div{Y}{\Div{X}{Z}} - h(Y, \Div{X}{Z}) + \Shape{h(X, Z)}{Y} - \tDiv{Y}{h(X, Z)} \\
	&- \Div{[X, Y]}{Z} - h([X, Y], Z) \\
	&= \Rm(X, Y)Z + \left(\Shape{h(X, Z)}{Y} - \Shape{h(Y, Z)}{X}\right) \\
	&+ \left(h(X, \Div{Y}{Z}) - h(Y, \Div{X}{Z}) - h([X, Y], Z) + \tDiv{X}{h(Y, Z)} - \tDiv{Y}{h(X, Z)}\right).
\end{align*}
最后一个等号中, 按上下两行分成两部分, 上一部分为切向, 而下一部分为法向. 可以用切/法向量与之内积得到以下公式:

\subsection{子流形基本方程·Gauss 方程}
取切向量场 $W \in \VecFld(M)$, 

\end{document}